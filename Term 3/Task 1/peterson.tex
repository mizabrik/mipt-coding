\documentclass[a4paper]{paper} 
\usepackage[margin=2.5cm]{geometry}
\usepackage{graphicx}
\usepackage{lipsum}
\usepackage{xcolor}
\usepackage{booktabs}
\usepackage[utf8]{inputenc}
\usepackage[russian]{babel}  
\usepackage{mathtools}
\usepackage{amssymb}
\usepackage{listings}
\sectionfont{\large\sf\bfseries\color{black!70!blue}} 
\title{Алгоритм петерсона}
\vspace{-2cm}
\author{Александр Васильев, 597} 

\renewcommand*{\epsilon}{\varepsilon}
\lstset{language=Pascal}

\begin{document} 
\maketitle

\begin{lstlisting}
def lock(t):
    victim = t
    want[t] = True

    while want[1 - t] and victim == t:
        pass
\end{lstlisting}
Данная реализация не гарантирует взаимного исключения в следующей ситуации
(начальное состояние --- \lstinline$victim = 0; want = [False, False]$):
\begin{enumerate}
\item Поток $1$: \lstinline$victim = 1$ (\lstinline$victim = 1; want = [False, False]$).
\item Поток $0$: \lstinline$victim = 0$ (\lstinline$victim = 0; want = [False, False]$).
\item Поток $0$: \lstinline$want[0] = True$ (\lstinline$victim = 0; want = [True, False]$).
\item Поток $0$: \lstinline$want[1] == False$, поток входит в критическую секцию.
\item Поток $1$: \lstinline$want[1] = True$ (\lstinline$victim = 0; want = [True, True]$).
\item Поток $1$: \lstinline$victim == 0$, поток входит в критическую секцию.
\end{enumerate}

\end{document}
