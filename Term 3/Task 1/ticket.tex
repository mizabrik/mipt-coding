\documentclass[a4paper]{paper} 
\usepackage[margin=2.5cm]{geometry}
\usepackage{graphicx}
\usepackage{lipsum}
\usepackage{xcolor}
\usepackage{booktabs}
\usepackage[utf8]{inputenc}
\usepackage[russian]{babel}  
\usepackage{mathtools}
\usepackage{amssymb}
\usepackage{listings}
\usepackage{hyperref}
\sectionfont{\large\sf\bfseries\color{black!70!blue}} 
\title{Ticket spinlock}
\vspace{-2cm}
\author{Александр Васильев, 597} 

\renewcommand*{\epsilon}{\varepsilon}
\lstset{language=C++}

\begin{document} 
\maketitle

Проблемы с произовдительностью начнутся, когда число потоков превысит
число потоков исполняющихся параллельно. Тогда может случиться, что spinlcok
будет разблокированн, но поток со следующим номером будет находиться в
конце очереди, тогда всем потокам придётся тратить своё время в ожидании до
того, как планировщик дойдёт до нашего избранника.

Test-and-set spinlock же не подвержен этой проблеме потому, что сразу после
разблокировки его займёт поток-везунчик, который исполняется на данный момент.
Цена честности --- это производительность :(
\end{document}
