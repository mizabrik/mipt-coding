\documentclass[a4paper]{paper} 
\usepackage[margin=2.5cm]{geometry}
\usepackage{graphicx}
\usepackage{lipsum}
\usepackage{xcolor}
\usepackage{booktabs}
\usepackage[utf8]{inputenc}
\usepackage[russian]{babel}  
\usepackage{mathtools}
\usepackage{amssymb}
\usepackage{listings}
\sectionfont{\large\sf\bfseries\color{black!70!blue}} 
\title{Идеальное паросочетание}
\vspace{-2cm}
\author{Александр Васильев, 597} 
\usepackage{mathtools}
\usepackage{amsthm}
\usepackage{amssymb}
\usepackage{amsfonts}
\usepackage{euscript}
\usepackage{mathalfa}
\usepackage{esvect}


\theoremstyle{plain}
\newtheorem{theorem}{Теорема}
\newtheorem{proposition}{Утверждение}
\newtheorem{lemma}{Лемма}
\newtheorem*{corollary}{Следствие}
\newtheorem*{fact}{Факт}

\theoremstyle{definition}
\newtheorem{definition}{Определение}

\renewcommand*{\epsilon}{\varepsilon}
\lstset{language=Pascal}

\newcommand{\To}{\Rightarrow}
\newcommand{\oT}{\Leftarrow}

\begin{document} 
\maketitle
\begin{definition}
Пусть дан граф $G = (V, E)$, $X \subseteq V$.
Тогда $N(X)$ --- это множество соседей вершин из $X$.
\end{definition}
\begin{theorem}[Холла]
Пусть $G = (V, E)$ — двудольный граф с долями $L$ и $R$,
тогда паросочетание размера $|L|$ существует тогда и только тогда,
когда для любого $A \subseteq L$, верно: $|A| \le |N(A)|$.
\end{theorem}
\begin{proof}~
\begin{description}
\item[$\Rightarrow$]
Существует паросочетание размера $|L| \To$ каждая вершина из $L$
соединена с какой-то другой $\To$ $|N(A)| \ge |A|$.
\item[$\Leftarrow$]
Заметим, что мы можем получить максимальное паросочетание с помощью нахождения
максимального потока: добавим исток $s$ и сток $t$, проведём рёбра $c = 1$
из $s$ в вершины $L$ и из $R$ в $t$, установим ребру $(u \in L, v \in R)$
пропускную способность $1$. Тогда насыщенные max flow рёбра
начального графа и представляют собой максимальное паросочетание:
по определению потока сумма по исходящим рёбрам из вершины $u \in L$
будет равна сумме по входящим, т. е. $\le 1$, симметрично для $v \in R$.
С другой стороны, аналогично, любое паросочетание задаёт поток, и его величина ---
это количество рёбер в паросочетании, а потому максимальному паросочетанию будет
соответствовать максимальный поток.
Итак, докажем от противного: пусть в $G$ нет идеального паросочетания, т. е.
$f_{\max} < |L|$. По теореме Форда-Фалкерсона, $f_{\max}$ равна величине
минимального разреза, т. е. существует разрез $(S, T)$ что
$cap(S,T) < |L|$. Пусть существует ребро
$(u, v)$ (в изначальном графе), что $u \in S$, $v \in T$.
Из двудольности следует, что $v \in R$, тогда добавив $v$ в $S$ мы
не увеличим $cap(S,T)$: из $v$ исходит только одно ребро $(v, t)$,
и его стоимость компенсируется устранением $(u, v)$ с разреза.
Итак, для любого ребра $(u, v)$ $u \in S \To v \in S$. Тогда, очевидно,
для $L_S = L \cap S$, $L_T = L \cap T$ и $R_S = R \cap S$ получаем
$cap(S, T) = |L_T| + |R_S| \le |L| = |L_S| + |L_T|$, т. е.
$|R_S| \le |L_S|$. Из того же утверждения о рёбрах следует, что
$N(L_S) \subseteq R_S$, т. е. $|N(L_S)| \le R_S \le L_S$.
\end{description}
\end{proof}
\end{document}
